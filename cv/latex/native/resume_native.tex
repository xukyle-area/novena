%!TEX program = xelatex
\documentclass[9pt,a4paper]{article}

% 基础包
\usepackage[left=1.2cm, right=1.2cm, top=0.8cm, bottom=0.8cm]{geometry}
\usepackage{xeCJK}
\usepackage{enumitem}
\usepackage{titlesec}
\usepackage{xcolor}
\usepackage{hyperref}

% 中文字体
\setCJKmainfont{PingFang SC}

% 颜色定义
\definecolor{titlecolor}{RGB}{0, 102, 204}
\definecolor{sectioncolor}{RGB}{51, 51, 51}

% 超链接设置
\hypersetup{
    colorlinks=true,
    linkcolor=titlecolor,
    urlcolor=titlecolor,
    pdfauthor={徐中建},
    pdftitle={徐中建 - 简历}
}

% 段落设置
\setlength{\parindent}{0pt}
\setlength{\parskip}{0pt}
\setlength{\lineskip}{0pt}
\setlength{\baselineskip}{10pt}
\pagestyle{empty}

% 自定义命令
\newcommand{\name}[1]{
    \begin{center}
        {\LARGE\bfseries\color{titlecolor} #1}
    \end{center}
    \vspace{1pt}
}

\newcommand{\contactinfo}[4]{
    \begin{center}
        {\small #1 \quad | \quad #2 \quad | \quad #3 \quad | \quad #4}
    \end{center}
    \vspace{3pt}
}

\newcommand{\sectiontitle}[1]{
    \vspace{3pt}
    {\large\bfseries\color{sectioncolor} #1}
    \vspace{1pt}
    \hrule height 0.8pt
    \vspace{2pt}
}

\newcommand{\jobentry}[5]{
    \textbf{#1} \hfill \textit{#2}\\
    \textit{#3} \hfill #4\\
    #5
    \vspace{2pt}
}

\newcommand{\projectentry}[3]{
    \textbf{#1}
    #3
    \vspace{3pt}
}

\begin{document}

% 姓名
\name{徐中建}

% 联系方式
\contactinfo{
    \href{mailto:zjxu97@gmail.com}{zjxu97@gmail.com}
}{
    +86 18971672214
}{
    \href{https://linkedin.com/in/xu-kyle}{linkedin.com/in/xu-kyle}
}{
    高级后端开发工程师 | 6年经验
}

% 教育经历
\sectiontitle{教育经历}

\textbf{黑龙江大学} \hfill \textit{2016.09 - 2020.06}\\
电子信息工程 本科

% 工作经历
\sectiontitle{工作经历}

\jobentry{高级软件开发工程师}{2023.05 - 至今}{老虎国际 · YAX 交易所(香港持牌交易所) · 交易后台组}{北京}{
负责加密货币交易所核心交易系统的设计与开发,构建实时行情系统与规则引擎平台
}

\jobentry{软件开发工程师}{2021.05 - 2023.03}{美团点评 · 到店事业群 · 酒店业务部 · 分销业务组}{北京}{
负责酒店分销商管理系统的开发与维护,实现接口限流与签名校验系统
}

\jobentry{软件开发工程师}{2019.10 - 2021.04}{小米集团 · 中国区 · 新零售业务部 · 系统开发组}{北京}{
参与新零售业务系统的开发与优化,负责订单系统与库存管理模块
}

% 核心项目经验
\sectiontitle{核心项目经验}

\projectentry{Flink 规则引擎系统}{Flink、Spring Boot、Kafka、Redis、MySQL、REST API}{
\begin{itemize}[leftmargin=0.8em, itemsep=1pt, parsep=0pt, topsep=0pt]
    \item \textbf{技术栈}: Flink、Spring Boot、Kafka、Redis、MySQL、REST API
    \item \textbf{项目背景}: 构建统一的实时流处理平台,支持动态任务配置与规则引擎,服务多业务场景
    \item \textbf{平台架构设计}: 构建多源数据接入架构,统一流式与批量任务处理,支持 Kafka、MQTT、WebSocket 等多种数据源
    \item \textbf{配置化管理}: 实现任务输入、逻辑、输出全链路配置化,支持业务方零代码快速上线新任务
    \item \textbf{动态规则引擎}: 支持 Flink SQL 规则动态提交、在线编辑与热加载,规则加载时间缩短 \textbf{60\%}
    \item \textbf{统一 API 体系}: 提供任务触发、监控、告警的 REST API,支持自动化运维
    \item \textbf{效率提升}: 任务上线周期从 \textbf{1 天降至 1 小时},支撑多业务线实时计算需求
    \item \textbf{技术亮点}: Flink 状态管理与 Checkpoint 机制保障 exactly-once 语义;基于 Spring Boot + Flink 的任务动态提交与生命周期管理
\end{itemize}
}

\projectentry{Crypto 交易所实时行情系统}{Flink、Kafka、Redis、DynamoDB、WebSocket、MQTT、Thrift}{
\begin{itemize}[leftmargin=0.8em, itemsep=1pt, parsep=0pt, topsep=0pt]
    \item \textbf{技术栈}: Flink、Kafka、Redis、DynamoDB、WebSocket、MQTT、Thrift
    \item \textbf{项目背景}: 统一内部撮合引擎与外部交易所的行情数据,为交易、风控、前端提供实时行情服务
    \item \textbf{内部行情流}: 对接撮合引擎,基于 Flink 实现窗口聚合与 K 线生成(1s/1m/5m/1h/1d),结果 sink 至 Kafka
    \item \textbf{外部行情聚合}: WebSocket 实时订阅 Binance、Crypto.com 等主流交易所,缓存与持久化
    \item \textbf{低延迟推送}: 通过 MQTT 协议实时推送行情,延迟降至 \textbf{< 500ms},支撑 \textbf{万级并发}
    \item \textbf{分层存储}: Redis 热数据缓存 + DynamoDB 历史数据持久化,支持监管报送
    \item \textbf{风控监控}: 实时监控价格剧烈波动、异常大单、流动性枯竭,触发预警并自动切换备用数据源
    \item \textbf{技术亮点}: Flink 滑动窗口聚合实现 K 线实时计算;分层缓存策略(L1: Guava Cache / L2: Redis)提升读性能
\end{itemize}
}

\projectentry{美团酒店分销商管理系统}{Spring、Redis、MySQL、XXL-Job、Thrift、State Machine、Guava Cache、MyBatis}{
\begin{itemize}[leftmargin=0.8em, itemsep=1pt, parsep=0pt, topsep=0pt]
    \item \textbf{技术栈}: Spring、Redis、MySQL、XXL-Job、Thrift、State Machine、Guava Cache、MyBatis
    \item \textbf{项目背景}: 面向全行业提供酒店门票分销服务,管理分销商全生命周期
    \item \textbf{接口限流系统}: 基于 Redis + Guava Cache 实现按商家维度的流量限制,防止恶意请求与系统过载
    \item \textbf{签名校验与鉴权}: Redis 存储签名信息,结合 Guava Cache 提高校验效率,QPS 提升 \textbf{3 倍}
    \item \textbf{生命周期管理}: State Machine 管理分销商状态流转(申请→审核→生效→暂停→下线),防止非法状态转换
    \item \textbf{商家画像与成本分析}: 处理商家打点信息,构建画像体系,支撑精准运营
    \item \textbf{技术亮点}: 滑动窗口限流算法(Redis + Lua 脚本);多级缓存架构(Guava Cache + Redis)降低 DB 压力
\end{itemize}
}

% 专业技能
\sectiontitle{专业技能与能力}

\begin{itemize}[leftmargin=0.8em, itemsep=3pt, parsep=0pt, topsep=0pt]
    \item \textbf{编程与框架}: 精通 Java 编程语言;熟练使用 Spring Boot、Spring Cloud、MyBatis 等后端框架
    \item \textbf{中间件}: 深度掌握 Kafka 消息队列、Redis 缓存、MySQL 数据库、DynamoDB、Thrift RPC 等
    \item \textbf{实时计算}: 深度掌握 Flink 流式计算,擅长设计万级并发、亚秒级延迟的分布式系统
    \item \textbf{架构设计}: 具备微服务架构、高并发系统设计、DDD 领域建模、状态机设计能力
    \item \textbf{运维工具}: 熟悉 Docker、Kubernetes、Prometheus、Grafana 等运维工具
    \item \textbf{业务能力}: 具备交易、结算、风控等金融系统全流程设计经验,能独立完成 0 到 1 架构设计
\end{itemize}

\end{document}
