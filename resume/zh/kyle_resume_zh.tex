%!TEX program = xelatex
\documentclass[10pt,a4paper]{article}

% 基础包
\usepackage[left=1.0cm, right=1.0cm, top=0.8cm, bottom=0.8cm]{geometry}
\usepackage{xeCJK}
\usepackage{enumitem}
\usepackage{titlesec}
\usepackage{xcolor}
\usepackage{hyperref}

% 中文字体
\setCJKmainfont{PingFang SC}

% 颜色定义
\definecolor{titlecolor}{RGB}{0, 102, 204}
\definecolor{sectioncolor}{RGB}{51, 51, 51}

% 超链接设置
\hypersetup{
    colorlinks=true,
    linkcolor=titlecolor,
    urlcolor=titlecolor,
    pdfauthor={徐中建},
    pdftitle={徐中建 - 简历}
}

% 段落设置
\setlength{\parindent}{0pt}
\setlength{\parskip}{0pt}
\setlength{\lineskip}{0pt}
\setlength{\baselineskip}{10pt}
\pagestyle{empty}

% 自定义命令
\newcommand{\name}[1]{
    \begin{center}
        {\LARGE\bfseries\color{titlecolor} #1}
    \end{center}
    \vspace{2pt}
}

\newcommand{\contactinfo}[4]{
    \begin{center}
        {\small #1 \quad | \quad #2 \quad | \quad #3 \quad | \quad #4}
    \end{center}
    \vspace{3pt}
}

\newcommand{\sectiontitle}[1]{
    \vspace{5pt}
    {\large\bfseries\color{sectioncolor} #1}
    \vspace{1pt}
    \hrule height 0.8pt
    \vspace{5pt}
}

\newcommand{\jobentry}[5]{
    \vspace{5pt}
    \textbf{#1} \hfill #2 \hfill \textit{#3}\\
    #5
    \vspace{2pt}
}

\newcommand{\projectentry}[3]{
    \vspace{5pt}
    {\large\textbf{#1}}
    \vspace{2pt}
    {\small\textit{#2}}
    \vspace{2pt}
    #3
    \vspace{5pt}
}

\begin{document}

% 姓名
\name{徐中建}

% 联系方式
\contactinfo{
    \href{mailto:zjxu97@gmail.com}{zjxu97@gmail.com}
}{
    +86 18971672214
}{
    \href{https://linkedin.com/in/xu-kyle}{linkedin.com/in/xu-kyle}
}{
    高级后端开发工程师( 6 年工作经验)
}

% 教育经历
\sectiontitle{教育经历}

\textbf{黑龙江大学} \hfill \textit{2016.09 - 2020.06}\\
电子信息工程 本科

% 工作经历
\sectiontitle{工作经历}

\jobentry{老虎证券 · 加密货币交易所 · 交易后台组}{高级软件工程师}{2023.05 - now}{北京}{
香港持牌交易所。负责交易所行情系统、OTC 业务和规则引擎平台的设计、开发与维护。
}

\jobentry{美团点评 · 到店事业群 · 酒店业务部 · 分销业务组}{软件工程师}{2021.05 - 2023.03}{北京}{
负责酒店分销商管理系统的开发与维护,实现接口限流与签名校验系统。
}

\jobentry{小米集团 · 中国区 · 新零售业务部 · 系统开发组}{软件工程师}{2019.10 - 2021.04}{北京}{
参与新零售业务系统的开发与优化,负责订单系统与库存管理模块。
}

% 核心项目经验
\sectiontitle{核心项目经验}

\projectentry{行情系统 \textit{@老虎证券}}{Flink, Kafka, Redis, DynamoDB, WebSocket, MQTT, Thrift}{
\begin{itemize}[leftmargin=0.8em, itemsep=1pt, parsep=0pt, topsep=0pt]
    \item \textbf{项目背景}: 构建统一的行情数据平台,整合内部撮合引擎与外部行情,为交易、风控和前端提供实时数据服务。
    \item \textbf{内部行情计算}: 对接撮合引擎输出的成交流,基于 Flink 实时计算 ticker、candle 和 orderbook 数据。
    \item \textbf{外部行情集成}: 通过 websocket 订阅外部主流交易所行情,集成外部行情到系统中,提高风险控制能力。
    \item \textbf{分层处理}: MQTT 输出实时数据、Redis 存储热数据、DynamoDB 持久化历史行情,满足实时推送,热点数据拉取,与历史数据查询。
    \item \textbf{低延迟推送}: 通过 MQTT 协议实时推送行情,端到端推送延迟时间小于 50ms。
\end{itemize}
}

\projectentry{规则引擎 \textit{@老虎证券}}{Flink, Spring Boot, Kafka, Redis, MySQL, REST API}{
\begin{itemize}[leftmargin=0.8em, itemsep=1pt, parsep=0pt, topsep=0pt]
    \item \textbf{项目背景}: 基于 Flink 设计了规则引擎平台,解决规则变更必须发版的问题,覆盖\textbf{风控、开户、KYC}等多业务场景。
    \item \textbf{规则配置中心}: 实现规则配置与SQL模板化、参数化管理,支持在线编辑与热加载,规则变更效率提升 \textbf{80\%}。
    \item \textbf{任务生命周期管理}: 实现任务动态提交与状态管理,支持 savepoint 重启与故障恢复,保障规则持续运行。
    \item \textbf{多源适配框架}: 采用适配器模式统一数据源与输出(Kafka、MySQL、WebSocket等),提升平台扩展性。
    \item \textbf{自动化运维}: 开发定时调度器监控规则变更并触发增量更新,任务上线周期从 1 天缩短至 1 小时。
\end{itemize}
}

\projectentry{OTC 系统 \textit{@老虎证券}}{Spring Boot, Redis, MySQL, XXL-Job, State Machine, Thrift, RateLimiter}{
\begin{itemize}[leftmargin=0.8em, itemsep=1pt, parsep=0pt, topsep=0pt]
    \item \textbf{项目背景}: 构建法币与加密货币的 OTC 交易系统,支持用户与 LP (流动性提供商) 大额兑换。
    \item \textbf{价格管理}: 自主维护订单簿,支持多 LP 接口与 Socket 双通道报价,聚合算法生成最优价格。
    \item \textbf{订单状态机}: 设计完整订单生命周期,覆盖用户的询价、下单、结算等流程。构建可靠的资金锁定与划转流程。
    \item \textbf{风控体系}: 基于 Flink 实时计算用户风险评分,结合开户信息与交易行为进行多维度风控。
    \item \textbf{结算优化}: 聚合未结算订单进行 Net 结算,减少划转次数,资金周转率提升 \textbf{70\%}。
\end{itemize}
}

\projectentry{分销商管理系统 \textit{@美团点评}}{Spring, Redis, MySQL, XXL-Job, Thrift, State Machine, Guava Cache, MyBatis}{
\begin{itemize}[leftmargin=0.8em, itemsep=1pt, parsep=0pt, topsep=0pt]
    \item \textbf{项目背景}: 美团酒旅面向全行业提供酒店门票分销服务,管理分销商全生命周期。
    \item \textbf{接口限流系统}: 基于 Redis 实现按商家与接口维度的流量限制,防止恶意请求与系统过载。
    \item \textbf{签名校验与鉴权}: Redis 存储签名信息,结合 Guava Cache 提高校验效率,QPS 提升 \textbf{3 倍}。
    \item \textbf{生命周期管理}: State Machine 管理分销商从申请,到接入,最终到下线的状态流转,防止非法状态转换。
    \item \textbf{商家画像与成本分析}: 处理商家打点信息,构建商家画像体系,为运营活动提供精准的数据支持。
    \item \textbf{接口限流}: 滑动窗口限流算法(Redis + Lua 脚本);多级缓存架构(Guava Cache + Redis)降低 database 压力。
\end{itemize}
}

% 专业技能
\sectiontitle{专业技能与能力}

\begin{itemize}[leftmargin=0.8em, itemsep=3pt, parsep=0pt, topsep=0pt]
    \item \textbf{编程与框架}: 精通 Java、熟悉 Python、SQL;熟练使用 Spring Boot、MyBatis 等后端框架。
    \item \textbf{中间件与存储}: 深度掌握 Flink 实时计算、Kafka 消息队列、Redis 缓存、MySQL 数据库、DynamoDB、Thrift RPC 等。
    \item \textbf{架构与运维}: 具备微服务架构、高并发系统设计、DDD 领域建模、状态机设计能力;熟悉 Docker、Kubernetes、Prometheus、Grafana 等运维工具。
    \item \textbf{实时计算与高并发架构}: 深度掌握 Flink 流式计算,擅长设计万级并发、亚秒级延迟的分布式系统,熟悉多级缓存与限流架构。
    \item \textbf{复杂业务系统设计}: 具备交易、结算、风控等金融系统全流程设计经验,擅长复杂业务抽象与状态机建模,能独立完成 0 到 1 架构设计。
    \item \textbf{平台化基础设施}: 构建可配置、可复用的规则引擎与任务调度平台,赋能业务快速迭代,任务上线周期从天级缩短至小时级。
\end{itemize}

\end{document}

