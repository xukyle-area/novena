%!TEX program = xelatex
\documentclass[10pt,a4paper]{article}

% 基础包
\usepackage[left=1.0cm, right=1.0cm, top=0.8cm, bottom=0.8cm]{geometry}
\usepackage{xeCJK}
\usepackage{enumitem}
\usepackage{titlesec}
\usepackage{xcolor}
\usepackage{hyperref}

% 中文字体
\setCJKmainfont{PingFang SC}

% 颜色定义
\definecolor{titlecolor}{RGB}{0, 102, 204}
\definecolor{sectioncolor}{RGB}{51, 51, 51}

% 超链接设置
\hypersetup{
    colorlinks=true,
    linkcolor=titlecolor,
    urlcolor=titlecolor,
    pdfauthor={徐中建},
    pdftitle={徐中建 - 简历}
}

% 段落设置
\setlength{\parindent}{0pt}
\setlength{\parskip}{0pt}
\setlength{\lineskip}{0pt}
\setlength{\baselineskip}{10pt}
\pagestyle{empty}

% 自定义命令
\newcommand{\name}[1]{
    \begin{center}
        {\LARGE\bfseries\color{titlecolor} #1}
    \end{center}
    \vspace{1pt}
}

\newcommand{\contactinfo}[4]{
    \begin{center}
        {\small #1 \quad | \quad #2 \quad | \quad #3 \quad | \quad #4}
    \end{center}
    \vspace{3pt}
}

\newcommand{\sectiontitle}[1]{
    \vspace{5pt}
    {\large\bfseries\color{sectioncolor} #1}
    \vspace{1pt}
    \hrule height 0.8pt
    \vspace{5pt}
}

\newcommand{\jobentry}[5]{
    \vspace{3pt}
    \textbf{#1} \hfill #2 \hfill \textit{#3}\\
    #5
    \vspace{2pt}
}

\newcommand{\projectentry}[3]{
    \vspace{3pt}
    {\large\textbf{#1}}
    \vspace{1pt}
    {\small\textit{#2}}
    \vspace{2pt}
    #3
    \vspace{3pt}
}

\begin{document}

% 姓名
\name{徐中建}

% 联系方式
\contactinfo{
    \href{mailto:zjxu97@gmail.com}{zjxu97@gmail.com}
}{
    +86 18971672214
}{
    \href{https://linkedin.com/in/xu-kyle}{linkedin.com/in/xu-kyle}
}{
    高级后端开发工程师( 6 年工作经验)
}

% 教育经历
\sectiontitle{教育经历}

\textbf{黑龙江大学} \hfill \textit{2016.09 - 2020.06}\\
电子信息工程 本科

% 工作经历
\sectiontitle{工作经历}

\jobentry{老虎证券 · 加密货币交易所 · 交易后台组}{高级软件工程师}{2023.05 - now}{北京}{
香港持牌交易所。负责交易所行情系统、OTC 业务和规则引擎平台的设计、开发与维护。
}

\jobentry{美团点评 · 到店事业群 · 酒店业务部 · 分销业务组}{软件工程师}{2021.05 - 2023.03}{北京}{
负责酒店分销商管理系统的开发与维护,实现接口限流与签名校验系统。
}

\jobentry{小米集团 · 中国区 · 新零售业务部 · 系统开发组}{软件工程师}{2019.10 - 2021.04}{北京}{
参与新零售业务系统的开发与优化,负责订单系统与库存管理模块。
}

% 核心项目经验
\sectiontitle{核心项目经验}

\projectentry{分布式风控决策平台 \textit{@老虎证券}}{Java, Flink, Spring Cloud, Kafka, Redis, MySQL, Docker, K8s}{
\begin{itemize}[leftmargin=0.8em, itemsep=1pt, parsep=0pt, topsep=0pt]
    \item \textbf{系统架构设计与技术攻坚}:主导从 0 到 1 规划分布式风控系统技术架构,完成规则引擎、实时计算框架、数据存储等核心技术选型,设计高并发、低延迟的决策平台。
    \item \textbf{基础设施建设}:搭建实时特征计算引擎、动态规则管理系统、风险画像平台与多策略决策工作流,支撑多业务场景的风险拦截。
    \item \textbf{数据体系与模型工程化}:构建风控专用数据仓库,设计实时/离线特征加工流水线,实现数据采集、清洗、存储到特征服务的全链路闭环。推动机器学习模型与规则策略融合,开发自动化模型部署与 AB 测试框架。
    \item \textbf{系统落地与性能优化}:主导核心模块编码,解决高并发流量下的性能瓶颈,系统可用性达 99.99\%。建立监控报警体系与容灾方案,实现风险拦截与系统稳定性的双重保障。
    \item \textbf{跨领域协同与标准化}:推动风控系统与支付、交易、用户中心等业务系统无缝集成,制定 API 接口规范与数据对接标准,沉淀系统建设方法论,培养团队风控技术能力。
\end{itemize}
}

\projectentry{行情系统 \textit{@老虎证券}}{Flink, Kafka, Redis, DynamoDB, WebSocket, MQTT, Thrift}{
\begin{itemize}[leftmargin=0.8em, itemsep=1pt, parsep=0pt, topsep=0pt]
    \item \textbf{项目背景}: 构建统一的行情数据平台,整合内部撮合引擎与外部交易所行情,为交易、风控和前端提供实时数据服务。
    \item \textbf{内部行情计算}: 对接撮合引擎的成交流,基于 Flink 滑动窗口实现 K 线实时计算,支持深度、成交量等指标聚合。
    \item \textbf{低延迟推送}: 基于 MQTT 协议实现行情实时推送,端到端延迟 \textbf{< 500ms},支撑 \textbf{万级并发订阅}。
    \item \textbf{外部行情接入}: 通过 WebSocket 实时订阅 Binance、Crypto.com 等主流交易所行情,实现数据标准化与缓存。
    \item \textbf{分层存储}: Redis 存储热数据 + DynamoDB 持久化历史行情,满足实时查询与监管报送需求。
    \item \textbf{风控监控}: 实时监控价格异常波动、大额成交、流动性枯竭等风险事件,触发预警并自动切换备用数据源。
    \item \textbf{技术亮点}: Flink 滑动窗口聚合与水位线机制保障数据一致性;多级缓存架构(Guava Cache + Redis)提高系统并发量。
\end{itemize}
}

\projectentry{OTC 系统 \textit{@老虎证券}}{Spring Boot, Redis, MySQL, XXL-Job, State Machine, Thrift, RateLimiter}{
\begin{itemize}[leftmargin=0.8em, itemsep=1pt, parsep=0pt, topsep=0pt]
    \item \textbf{项目背景}: 构建法币与加密货币的 OTC 交易系统,支持用户与 LP(流动性提供商)大额兑换。
    \item \textbf{价格管理}: 自主维护订单簿,支持多 LP 接口与 Socket 双通道报价,聚合算法生成最优价格。
    \item \textbf{订单状态机}: 设计完整订单生命周期(报价→锁定→成交→结算→确认),覆盖资产锁定与资金划转。
    \item \textbf{风控体系}: 基于 Flink 实时计算用户风险评分,结合开户信息与交易行为进行多维度风控。
    \item \textbf{Net 结算优化}: 聚合未结算订单进行 Net 结算,减少划转次数,资金周转率提升 \textbf{70\%}。
\end{itemize}
}

\projectentry{分销商管理系统 \textit{@美团点评}}{Spring, Redis, MySQL, XXL-Job, Thrift, State Machine, Guava Cache, MyBatis}{
\begin{itemize}[leftmargin=0.8em, itemsep=1pt, parsep=0pt, topsep=0pt]
    \item \textbf{项目背景}: 面向全行业提供酒店门票分销服务,管理分销商全生命周期。
    \item \textbf{接口限流系统}: 基于 Redis + Guava Cache 实现按商家维度的流量限制,防止恶意请求与系统过载。
    \item \textbf{签名校验与鉴权}: Redis 存储签名信息,结合 Guava Cache 提高校验效率,QPS 提升 \textbf{3 倍}。
    \item \textbf{生命周期管理}: State Machine 管理分销商状态流转(申请→审核→生效→暂停→下线),防止非法状态转换。
    \item \textbf{商家画像与成本分析}: 处理商家打点信息,构建画像体系,支撑精准运营。
    \item \textbf{接口限流}: 滑动窗口限流算法(Redis + Lua 脚本);多级缓存架构(Guava Cache + Redis)降低 database 压力。
\end{itemize}
}

% 专业技能
\sectiontitle{专业技能与能力}

\begin{itemize}[leftmargin=0.8em, itemsep=3pt, parsep=0pt, topsep=0pt]
    \item \textbf{系统架构与分布式设计}:精通高并发、低延迟分布式系统架构,具备从 0 到 1 规划风控平台、规则引擎、实时计算引擎的实战经验。
    \item \textbf{编程与云原生}:精通 Java,熟悉 Spring Cloud/Dubbo/Gin 等微服务框架,掌握 Docker、Kubernetes、Service Mesh 等云原生技术,具备容器化部署与云运维能力。
    \item \textbf{大数据与实时计算}:熟练使用 Flink/Spark Streaming 构建实时特征计算引擎,掌握 Kafka/Pulsar 消息队列,熟悉 MySQL/PostgreSQL、HBase/Redis/ES、ClickHouse/Doris 等海量数据存储与 OLAP 技术。
    \item \textbf{风控与模型工程化}:熟悉 Drools/Aviator 等规则引擎、Camunda/自研决策流引擎开发与优化,具备特征工程、实时特征服务、特征监控全流程经验。了解 TensorFlow Serving/MLflow 等模型部署与 AB 测试框架。
    \item \textbf{系统稳定性与性能优化}:擅长高并发系统性能调优、容灾设计与监控报警体系建设,保障系统 99.99\% 可用性。
    \item \textbf{跨团队协作与标准化}:推动风控系统与业务系统深度集成,制定接口与数据标准,注重技术文档与团队能力建设。
\end{itemize}

\end{document}

